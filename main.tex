\documentclass[a4paper,12pt]{article}

\usepackage[utf8]{inputenc}
\usepackage[T1]{fontenc}
\usepackage[slovak]{babel}
\usepackage{graphicx}
\usepackage{hyperref}
\usepackage{indentfirst}
\usepackage{geometry}
\geometry{margin=2.5cm}

\title{
    Slovenská technická univerzita v Bratislave\\
    Fakulta informatiky a informačných technológií\\[1em]
    \large MIP\\[2em]
    \textbf{Online kalendár pre plánovanie a organizáciu času}
}
\author{Anton Hrimov}
\date{September 2025}

\begin{document}

\maketitle
\thispagestyle{empty}

\vspace{1cm}
\textbf{Krátky popis projektu:} 
Analýza a návrh online kalendára, ktorý poskytuje široké možnosti pre organizáciu času, plánovanie stretnutí, sledovanie aktivít a zdieľanie plánov medzi používateľmi. Projekt je zameraný na vytvorenie moderného a praktického nástroja, ktorý pomôže študentom a aktívnym ľuďom efektívnejšie plánovať svoj čas.

\newpage

\section{Introduction}
V dnešnej dobe sa efektívne plánovanie času stáva čoraz dôležitejším. Študenti, profesionáli aj bežní používatelia často hľadajú spôsoby, ako zladiť školské, pracovné a osobné povinnosti. Cieľom tohto projektu je vytvoriť moderný online kalendár, ktorý zjednoduší organizáciu času a poskytne používateľom možnosť zdieľať svoje plány s ostatnými.

\section{Goals of the Project}
Hlavným cieľom projektu je vyvinúť interaktívny online systém, ktorý umožní:
\begin{itemize}
    \item vytvárať a upravovať udalosti v kalendári,
    \item organizovať stretnutia s ostatnými používateľmi,
    \item prezerať si aktivity a plány priateľov alebo kolegov,
    \item plánovať nielen pracovné úlohy, ale aj osobné voľnočasové aktivity,
    \item nastaviť pripomienky na dôležité udalosti.
\end{itemize}

\section{Implementation Overview}
Systém bude navrhnutý ako webová aplikácia. Používateľské rozhranie bude intuitívne, s možnosťou vizualizácie týždenného alebo mesačného prehľadu.  
Backend bude zabezpečovať spracovanie používateľských údajov, plánovanie udalostí a synchronizáciu kalendárov medzi používateľmi.  
Projekt je možné rozšíriť o:
\begin{itemize}
    \item synchronizáciu s Google Calendar alebo mobilnými zariadeniami,
    \item AI odporúčania pre efektívne plánovanie,
    \item analytiku využitia času.
\end{itemize}

\section{Conclusion}
Navrhnutý online kalendár poskytuje moderné riešenie pre študentov aj profesionálov, ktorí chcú mať svoj čas pod kontrolou.  
Cieľom je spojiť jednoduchosť používania s vysokou mierou funkcionality a sociálnej interakcie, aby sa plánovanie stalo prirodzenou a príjemnou súčasťou každodenného života.

\newpage
\begin{thebibliography}{9}
\bibitem{coplien1999}
J. O. Coplien. \textit{Multi-Paradigm Design for C++}. Addison-Wesley, 1999.

\bibitem{czarnecki2005}
K. Czarnecki, S. Helsen, and U. Eisenecker.
Staged configuration through specialization and multi-level configuration of feature models.
\textit{Software Process: Improvement and Practice}, 10:143–169, Apr./June 2005.

\bibitem{czarneckiKim2005}
K. Czarnecki and C. H. P. Kim.
Cardinality-based feature modeling and constraints: A progress report.
In \textit{International Workshop on Software Factories}, OOPSLA 2005, San Diego, USA, Oct. 2005.

\bibitem{kang1990}
K. C. Kang, S. G. Cohen, J. A. Hess, W. E. Novak, and A. S. Peterson.
Feature-oriented domain analysis (FODA): A feasibility study.
Technical Report CMU/SEI-90-TR-21, Software Engineering Institute,
Carnegie Mellon University, Pittsburgh, USA, Nov. 1990.

\bibitem{sei2005}
C. M. U. Software Engineering Institute.
A framework for software product line practice—version 5.0.
\url{http://www.sei.cmu.edu/productlines/frame_report/}
\end{thebibliography}

\end{document}

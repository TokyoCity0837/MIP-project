\documentclass[a4paper,12pt]{article}

\usepackage[utf8]{inputenc}
\usepackage[T1]{fontenc}
\usepackage[slovak]{babel}
\usepackage{graphicx}
\usepackage{hyperref}
\usepackage{indentfirst}
\usepackage{geometry}
\geometry{margin=2.5cm}

\title{
    Slovenská technická univerzita v Bratislave\\
    Fakulta informatiky a informačných technológií\\[1em]
    \large MIP\\[2em]
    \textbf{Online kalendár pre plánovanie a organizáciu času}
}
\author{Anton Hrimov}
\date{September 2025}

\begin{document}

\maketitle
\thispagestyle{empty}

\vspace{1cm}
\textbf{Krátky popis projektu:}  
Analýza a návrh online kalendára, ktorý poskytuje široké možnosti pre organizáciu času, plánovanie stretnutí, sledovanie aktivít a zdieľanie plánov medzi používateľmi. Projekt je zameraný na vytvorenie moderného a praktického nástroja, ktorý pomôže študentom a aktívnym ľuďom efektívnejšie plánovať svoj čas.

\newpage

\section{Úvod}
V dnešnej dobe sa efektívne plánovanie času stáva čoraz dôležitejším. Študenti, profesionáli aj bežní používatelia často hľadajú spôsoby, ako zladiť školské, pracovné a osobné povinnosti.  
Cieľom tohto projektu je vytvoriť moderný online kalendár, ktorý zjednoduší organizáciu času a poskytne používateľom možnosť zdieľať svoje plány s ostatnými \cite{newport_deepwork}.

\section{Ciele projektu}
Hlavným cieľom projektu je vyvinúť interaktívny online systém, ktorý umožní:
\begin{itemize}
    \item vytvárať a upravovať udalosti v kalendári,
    \item organizovať stretnutia s ostatnými používateľmi,
    \item prezerať si aktivity a plány priateľov alebo kolegov,
    \item plánovať nielen pracovné úlohy, ale aj osobné voľnočasové aktivity,
    \item nastaviť pripomienky na dôležité udalosti.
\end{itemize}

\section{Prehľad implementácie}
Systém bude navrhnutý ako webová aplikácia. Používateľské rozhranie bude intuitívne, s možnosťou vizualizácie týždenného alebo mesačného prehľadu \cite{nielsen_usability}.  
Backend bude zabezpečovať spracovanie používateľských údajov, plánovanie udalostí a synchronizáciu kalendárov medzi používateľmi.  
Projekt je možné rozšíriť o:
\begin{itemize}
    \item synchronizáciu s Google Calendar alebo mobilnými zariadeniami \cite{google_calendarapi},
    \item AI odporúčania pre efektívne plánovanie,
    \item analytiku využitia času.
\end{itemize}

\section{Konkurenčné výhody a cielenie na študentov}

Navrhovaný online kalendár prináša niekoľko kľúčových výhod, ktoré ho odlišujú od existujúcich riešení na trhu:

\begin{itemize}
    \item \textbf{Jednoduchosť a intuitívnosť používania:} Používateľské rozhranie je navrhnuté tak, aby aj menej skúsení používatelia, vrátane školákov, dokázali rýchlo vytvárať, upravovať a spravovať svoje plány.
    \item \textbf{Sociálna interakcia:} Kalendár umožňuje zdieľať plány s kamarátmi, spolužiakmi a rodinou, čo podporuje tímovú spoluprácu pri školských projektoch alebo skupinových aktivitách.
    \item \textbf{Personalizované pripomienky a notifikácie:} Používatelia dostávajú inteligentné upozornenia, ktoré im pomáhajú efektívne zvládať úlohy a nezabudnúť na dôležité termíny, napr. odovzdanie domácich úloh či skúšok.
    \item \textbf{AI odporúčania pre študentov:} Systém dokáže analyzovať pracovný a voľnočasový plán používateľa a navrhovať optimálne rozloženie aktivít tak, aby sa minimalizovalo preťažovanie a zlepšila produktivita.
    \item \textbf{Prepojenie so školskými a študijnými systémami:} Kalendár môže byť integrovaný s rozvrhmi, online učebnicami a digitálnymi platformami, čo uľahčuje študentom sledovať školské povinnosti a termíny.
\end{itemize}

Vďaka týmto vlastnostiam je náš online kalendár špeciálne vhodný pre \textbf{študentov a školákov}, ktorí potrebujú nielen plánovať svoje úlohy, ale aj efektívne kombinovať školské povinnosti s voľnočasovými aktivitami. Projekt tak poskytuje moderné, flexibilné a praktické riešenie, ktoré podporuje produktívny a organizovaný život mladých používateľov.

\section{Záver}
Navrhnutý online kalendár poskytuje moderné riešenie pre študentov aj profesionálov, ktorí chcú mať svoj čas pod kontrolou.  
Cieľom je spojiť jednoduchosť používania s vysokou mierou funkcionality a sociálnej interakcie, aby sa plánovanie stalo prirodzenou a príjemnou súčasťou každodenného života \cite{norman_design}.

\newpage
\begin{thebibliography}{9}

\bibitem{nielsen_usability}
Nielsen, J. (1994). \textit{Usability Engineering}. Morgan Kaufmann.

\bibitem{norman_design}
Norman, D. (2013). \textit{The Design of Everyday Things (Revised and Expanded Edition)}. Basic Books.

\bibitem{cagan_inspired}
Cagan, M. (2017). \textit{Inspired: How to Create Tech Products Customers Love}. Wiley.

\bibitem{mcconnell_code}
McConnell, S. (2004). \textit{Code Complete: A Practical Handbook of Software Construction}. Microsoft Press.

\bibitem{pressman_swebok}
Pressman, R. S., \& Maxim, B. R. (2019). \textit{Software Engineering: A Practitioner's Approach (9th Edition)}. McGraw-Hill Education.

\bibitem{newport_deepwork}
Newport, C. (2016). \textit{Deep Work: Rules for Focused Success in a Distracted World}. Grand Central Publishing.

\bibitem{google_calendarapi}
Google Developers. (2024). \textit{Google Calendar API Documentation}. Retrieved from \url{https://developers.google.com/calendar}

\bibitem{huang_calendarux}
Huang, L., \& Zhang, M. (2022). UX Design Patterns in Digital Calendar Applications. \textit{Proceedings of the International Conference on Human-Computer Interaction}, Springer.

\bibitem{chen_calendar}
Chen, Y., \& Li, Q. (2021). Design and Implementation of an Online Calendar System Based on Cloud Services. \textit{International Journal of Computer Applications}, 183(45), 1–5.

\end{thebibliography}

\end{document}
